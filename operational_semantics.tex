\documentclass[12pt,letterpaper]{article}
\usepackage{geometry}
\geometry{margin=1 in}
\usepackage[T1]{fontenc}
\usepackage[]{amsmath}
\usepackage{longtable}
\usepackage{semantic}
\title{crumbL}
\author{Raymond Chee, Chris Denny, Mark Ebbole, Carolyn Tran, Sarah Wang, Angela Xie}
\date{Thursday December 1}
\begin{document}

% introduction section about what we're trying to accomplish (make L imperative with type inferencing)
% also have a section about types
% for each section, explain why we chose to do certain things that are different from L



\section{Operational Semantics}

\subsection{Environments}

$E$ - a mapping from identifiers to values

$F$ - a mapping from identifiers to function definitions. 

A function definition has three fields: 

prog: a statement of the form $P$ (see context free grammar)

r: a statement of the form $e$ that comes after the ret keyword.

p\_list: a list of identifiers that make up the parameter list of the function definition.


\subsection{The Fundamental crumbL Statement}
\inference{E,F |- S_1 : E^\prime, F^\prime \\
		  E^\prime, F^\prime |- S_2 : E^{\prime\prime}, F^{\prime\prime}}
		  {E,F |- S_1\ S_2 : E^{\prime\prime},F^{\prime\prime}}

\ 

\ 


\subsection{Constants}



\inference{\text{integer i}}{E,F |- \text{i} : \text{i} }

\


\inference{\text{String s}}{E,F |- \text{s} : \text{s} }

\ 

\inference{}{E,F |- \text{Nil} : \text{Nil} }



\subsection{Arithmetic}


\inference{E,F |- e_1 : i_1 \\
           E,F |- e_2 : i_2}
		   {E,F |- e_1 \oplus e_2 : i_1 \oplus i_2 }

\
		   
where $\oplus \in \{+, -, *, \%\}$

\

\

\inference{E,F |- e_1 : i_1 \\
           E,F |- e_2 : i_2 \\
           i_2 \neq 0}
		   {E,F |- e_1 / e_2 : i_1 / i_2 }
		   
\

\

\inference{E,F |- e_1 : s_1 \\
           E,F |- e_2 : s_2}
		   {E,F |- e_1::e_2 : s_1s_2 }
		   
		   
\subsection{Lists}
		   
\inference{E,F |- e_1 : v_1 \text{(not a list)}\\
           E,F |- e_2 : v_2 \ \text{(not Nil)}}
		   {E,F |- e_1@e_2 : [v_1, v_2] }
		   

\

\

\inference{E,F |- e_1 : v_1 \text{(not a list)} \\
           E,F |- e_2 : \text{Nil} }
		   {E,F |- e_1@e_2 : v_1 }
		   
\

\

\inference{E,F |- e : [v_1, v_2] }
		   {E,F |- !e : v_1 }
		   
		   
\

\

\inference{E,F |- e : [v_1, v_2] }
		   {E,F |- \#e : v_2 }
		   
\

\

\inference{E,F |- e : v_1 \text{(not a list)} }
		   {E,F |- !e : v_1 }
		   
		   
\

\

\inference{E,F |- e : v_1 \text{(not a list)} }
		   {E,F |- \#e : Nil }
		   
	
	

\subsection{Boolean Logic}


\

\

\inference{E,F |- e_1 : v_1 \\
           E,F |- e_2 : v_2}
           {E,F |- e_1 \odot e_2 : v_1 \odot v_2}
           
where $v_1$ and $v_2$ are either both strings or both integers. If strings, comparisons are lexicgraphic.

$\odot \in \{<, >, < =, >=, ==, != \}$



\

\

Let \textbf{False} be "", 0, or Nil.

\textbf{True} is any expression that evaluates to something that is not \textbf{False}.

\

\

\inference{E,F |- e_1 : \text{True} \\
		   E,F |- e_2 : \text{True}}
		   {E,F |- e_1 \text{ and } e_2 : 1}
		   
		   
\

\

\inference{E,F |- e_1 : \text{False}}
		   {E,F |- e_1 \text{ and } e_2 : 0}
		   
		   
\

\

\inference{E,F |- e_1 : \text{True} \\
		   E,F |- e_2 : \text{False}}
		   {E,F |- e_1 \text{ and } e_2 : 0}
		   
		   
		   
		   



\

\

\inference{E,F |- e_1 : \text{True}}
		   {E,F |- e_1 \text{ or } e_2 : 1}
		   
		   
\

\

\inference{E,F |- e_1 : \text{False} \\
           E,F |- e_2 : \text{False}}
		   {E,F |- e_1 \text{ or } e_2 : 0}
		   
		   
\

\

\inference{E,F |- e_1 : \text{False} \\
           E,F |- e_2 : \text{True}}
		   {E,F |- e_1 \text{ or } e_2 : 1}




\

\


\inference{E,F |- e_1 : \text{True}}
           {E,F |- \text{not } e_1 : 0}
           
\

\

   
\inference{E,F |- e_1 : \text{False}}
           {E,F |- \text{not } e_1 : 1}
		   
\

\



\inference{E,F |- e_1 : Nil}
           {E,F |- \text{isNil } e_1 : 1}
		   
\

\


\inference{E,F |- e_1 : \text{not Nil}}
           {E,F |- \text{isNil } e_1 : 0}
		   
\

\


\subsection{Conditional Statements}



\inference{E,F |- C: \text{True}  \\
		   E,F |- S_1 : E^\prime, F^\prime}
		   {E,F |- \text{if }(C) \text{ then } S_1 \text{ else } S_2 \text{ fi } : E^\prime, F^\prime}
		   
\ 

\ 

\inference{E,F |- C: \text{False}  \\
		   E,F |- S_2 : E^\prime, F^\prime}
		   {E,F |- \text{if }(C) \text{ then } S_1 \text{ else } S_2 \text{ fi } : E^\prime, F^\prime}
		   
\ 

\ 

\inference{E,F |- C: \text{True}  \\
		   E,F |- S : E^\prime, F^\prime}
		   {E,F |- \text{if }(C) \text{ then } S \text{ fi } : E^\prime, F^\prime}
		   
\ 

\ 

\inference{E,F |- C: \text{False}}
		   {E,F |- \text{if }(C) \text{ then } S \text{ fi } : E, F}
		   
\ 

\ 


\inference{E,F |- C: \text{False} }
		   {E,F |- \text{while}(C) \text{ do } S \text{ ob } : E, F}
		   
		   
\

\
 


\inference{E,F |- C: \text{True} \\
		   E,F |- S : E^\prime, F \\
		   E^\prime, F |- \text{while}(C) \text{ do } S \text{ ob } : E^{\prime\prime}, F}
		   {E,F |- {while}(C) \text{ do } S \text{ ob } : E^{\prime\prime}, F}	  
		
\
   
Note: Under this rule, the statement S must not change the function environment.


\ 

\

\subsection{Identifiers and Functions}

\inference{E,F |- e: v}
		   {E,F |- id = e; : E[id \leftarrow v], F}
		   
		   
\ 

\ 


\inference{}{E,F |- \text{lazy } id = e; : E[id \leftarrow e], F}
		   
		   
\ 

\ 

\inference{e = E[id] \\
           E,F |- e : v \\
           E^\prime = E[id \leftarrow v]}
		   {E,F |- id : v, E^\prime, F}
		   
\

\

		   
\inference{fentry = \{\text{prog: P, r: e, p\_list: p\_list} \} \\
		   F^\prime = F[\textbf{fname} \leftarrow fentry]  }
{E,F |- \text{func } \textbf{fname}(\text{p\_list})\ P\ \text{ret } e;\ \text{cnuf } : E, F^\prime}
		   
\

\

\inference{fentry = F[\textbf{fname}] \\
           E^\prime = \text{apply}(fentry.\text{p\_list},\ \text{call\_list}) \\
           p = fentry.\text{prog} \\
           E^\prime, F |- p : E^{\prime\prime} \\
           E^{\prime\prime}, F |- fentry.\text{r} : v }
{E,F |- \textbf{fname}(\text{call\_list}) : v}


\

Note: apply is just an operational semantics subroutine to construct a new environment for a called function, and is not useable from source code.


\

\inference{E,F |- \text{p\_list} = [p_1, R_1] \\
           E,F |- \text{call\_list} = [e_1, R_2] \\
           E,F |- e_1 : v_1 \\
           E,F |- \text{apply}(R_1, R_2) : E^\prime \\
           E^{\prime\prime} = E^\prime[p_1 \leftarrow v_1]}
           {E,F |- \text{apply}(\text{p\_list}, \text{call\_list}) : E^{\prime\prime}}
           
           
\

\

\inference{E,F |- \text{p\_list} = [\text{lazy } p_1, R_1] \\
           E,F |- \text{call\_list} = [e_1, R_2] \\
           E,F |- \text{apply}(R_1, R_2) : E^\prime \\
           E^{\prime\prime} = E^\prime[p_1 \leftarrow e_1]}
           {E,F |- \text{apply}(\text{p\_list}, \text{call\_list}) : E^{\prime\prime}}
           
           
\

\

\inference{}{\text{apply}(\epsilon, \epsilon) : \emptyset}
		   
		   
		   
\subsection{I/O}


\inference{E,F |- e: v}
		   {E,F |- \text{print}(e); : E,F, \text{ print out }v }
		   
\

\



\inference{T |- e_1: \alpha_1 \\
           T |- e_2: \alpha_2 \\
           \alpha_2 = \alpha_1 \text{ or } \alpha_2 = \alpha_1List}
		   {T |- e_1@e_2 : \alpha_1List}
		   
\

\


\end{document}
